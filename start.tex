\documentclass[10pt,fleqn]{article}
\usepackage[textwidth=16cm, textheight=24cm]{geometry}

\usepackage{graphicx}

% \usepackage[condensed,math]{iwona}
\usepackage[russian,english]{babel}
\usepackage[T2C]{fontenc}
% \usepackage{uarial}
% \renewcommand{\familydefault}{\sfdefault}
% \usepackage{blindtext}

\usepackage{array}
\usepackage{tabularx}
\usepackage{ragged2e}
\usepackage{amsmath}
\usepackage{enumitem}
\usepackage{longtable}

% For russian controls
\usepackage{amssymb}

% Global vars
\newcommand\formulaGap{9.3cm}
\newcommand\reWidth{3cm}
\newcommand\rightCeilWidth{3.7cm}

\title{Степени тригонометрических функций и линейных функций}
\author{Илья Семенов}

\pagenumbering{gobble}

% Makes possible using vars
\makeatletter

% Document margins correction
\sloppy

% Adds new table column styles
\newcolumntype{C}[1]{>{\centering\let\newline\\\arraybackslash\hspace{0pt}}m{#1}}
\newcolumntype{R}[1]{>{\RaggedLeft\arraybackslash}p{#1}}

\setlength{\tabcolsep}{0pt}
\renewcommand{\arraystretch}{1}
\renewcommand{\leq}{\leqslant}
\renewcommand{\geq}{\geqslant}

\begin{document}
	\begin{tabularx}{\textwidth}{m{.1\textwidth} C{.8\textwidth} R{.1\textwidth}}
		3.631 & \@title & \textbf{401} \\ 
		\hline
		\hline
	\end{tabularx}

	\begin{itemize}[noitemsep,topsep=0pt]
		\item [7.]
		\begin{tabular}{C{.8\textwidth} R{.2\textwidth}}
			\vbox{
				\[
					\int_{0}^{\pi} x\cos(2m+1)x \,dx = 0
				\]
			} & ГХ2 (332)(12a)
		\end{tabular}
		\item [8.]
		\begin{tabular}{C{\formulaGap} l r}
			\multicolumn{3}{c}{\vbox{\[
				\int_{0}^{\pi} \sin^{v-1}x\cos ax \,dx = \frac{\pi \cos\frac{a \pi}{2}}{2^{\nu-1}\nu B\left(\dfrac{\nu+a+1}{2},\dfrac{\nu-a+1}{2}\right)}
			\]}} \\
			 & [$Re\enspace\nu > 0$] & Лб V 121(68)и, Вт 337и
		\end{tabular}
		\item [9.]
		\begin{tabular}{C{\formulaGap} l r}
			\multicolumn{3}{c}{\vbox{\[
				\int_{0}^{\pi/2} \sin^{v-1}x\cos ax \,dx = \frac{\pi}{2^{\nu}\nu B\left(\dfrac{\nu+a+1}{2},\dfrac{\nu-a+1}{2}\right)}
			\]}} \\
			 & [$Re\enspace\nu > 0$] & ГХ2 (332)(9c)
		\end{tabular}
		\item [10.]
		\begin{tabular}{C{\formulaGap} m{\reWidth} R{\rightCeilWidth}}
			\[
				\int_{0}^{\pi/2} \sin^{\nu-2}x\cos\nu x \,dx = \frac{1}{\nu-1}\sin\frac{\nu\pi}{2}
			\] & [$Re\enspace\nu > 1$] & Х2 (332)(9c)
		\end{tabular}
		\item [11.]
		\begin{tabular}{C{\formulaGap} m{\reWidth} R{\rightCeilWidth}}
			\[\int_{0}^{\pi} \sin^{\nu}x\cos\nu x \,dx = \frac{\pi}{2^{\nu}}\cos\frac{\nu\pi}{2}\] 
			& 
			[$Re\enspace\nu > -1$] & Лб V 121(70)и
		\end{tabular}
		\item [12.]
			\begin{align*}
				\int_{0}^{\pi} \sin^{2n}x\cos2mx \,dx & = 2\int_{0}^{\pi/2} \sin^{2n}x\cos2mx \,dx = \frac{(-1)^{m}}{2^{2n}}\begin{pmatrix}
					2n \\
					n-m
				\end{pmatrix} \pi \tag*{[$n \geq m$]} \\
				& = 0 \tag*{[$n < m$]} 
			\end{align*}
			\begin{flushright}
				Би (40)(16), ГХ2 (332)(12b)
			\end{flushright}
	\end{itemize}
\end{document}