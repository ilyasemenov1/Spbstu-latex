\documentclass[10pt,fleqn]{article}
\usepackage[textwidth=18cm, textheight=24cm]{geometry}

\usepackage{graphicx}

% \usepackage[condensed,math]{iwona}
\usepackage[russian,english]{babel}
\usepackage[T2C]{fontenc}

\usepackage{array}
\usepackage{tabularx}
\usepackage{ragged2e}
\usepackage{amsmath}
\usepackage{enumitem}
\usepackage{longtable}

% For russian controls
\usepackage{amssymb}

% Global vars
\newcommand\formulaGap{9.3cm}
\newcommand\reWidth{3cm}
\newcommand\rightCeilWidth{3.7cm}

\title{Степени тригонометрических функций и линейных функций}
\author{Илья Семенов}

\pagenumbering{gobble}

% Makes possible using vars
\makeatletter

% Document margins correction
\sloppy

% Add new table column styles
\newcolumntype{C}[1]{>{\centering\let\newline\\\arraybackslash\hspace{0pt}}m{#1}}
\newcolumntype{R}[1]{>{\RaggedLeft\arraybackslash}p{#1}}

\setlength{\tabcolsep}{0pt}
\renewcommand{\arraystretch}{1}

% Replace eng >= <= to rus ones
\renewcommand{\leq}{\leqslant}
\renewcommand{\geq}{\geqslant}

% Remove space around align
\setlength{\abovedisplayskip}{0pt}
\setlength{\belowdisplayskip}{0pt}
\setlength{\abovedisplayshortskip}{0pt}
\setlength{\belowdisplayshortskip}{0pt}

\begin{document}
	\begin{tabularx}{\textwidth}{m{.1\textwidth} C{.8\textwidth} R{.1\textwidth}}
		3.631 & \@title & \textbf{401} \\ 
		\hline
		\hline
	\end{tabularx}

	
	\begin{align*}
		\makebox[0pt]{7.}  \qquad \quad \int_{0}^{\pi} x\cos(2m+1)x \,dx = 0 \tag*{ ГХ2 (332)(12a)}
	\end{align*}
	
	\begin{align*}
		\makebox[0pt]{8.} \qquad \quad \int_{0}^{\pi} \sin^{v-1}x\cos ax \,dx = \frac{\pi \cos\frac{a \pi}{2}}{2^{\nu-1}\nu B\left(\dfrac{\nu+a+1}{2},\dfrac{\nu-a+1}{2}\right)} & \\
		& [Re\enspace\nu > 0] \tag*{Лб V 121(68)и, Вт 337и}
	\end{align*}
	\begin{align*}
		\makebox[0pt]{9.} \qquad \quad \int_{0}^{\pi/2} \sin^{v-1}x\cos ax \,dx = \frac{\pi}{2^{\nu}\nu B\left(\dfrac{\nu+a+1}{2},\dfrac{\nu-a+1}{2}\right)} & \\
		& [Re\enspace\nu > 0] \tag*{ГХ2 (332)(9c)}
	\end{align*}
	\begin{align*}
		\makebox[0pt]{10.} \qquad \quad \int_{0}^{\pi/2} \sin^{\nu-2}x\cos\nu x \,dx = \frac{1}{\nu-1}\sin\frac{\nu\pi}{2} &
		& [Re\enspace\nu > 1] \hspace{1cm} \tag*{Х2 (332)(9c)}
	\end{align*}
	\begin{align*}
		\makebox[0pt]{11.} \qquad \quad \int_{0}^{\pi} \sin^{\nu}x\cos\nu x \,dx = \frac{\pi}{2^{\nu}}\cos\frac{\nu\pi}{2} &
		\hspace{.7cm} & [Re\enspace\nu > -1] \tag*{Лб V 121(70)и}
	\end{align*}
	\begin{align*}
		\makebox[0pt]{12.} \qquad \quad \int_{0}^{\pi} \sin^{2n}x\cos2mx \,dx & = 2\int_{0}^{\pi/2} \sin^{2n}x\cos2mx \,dx = \frac{(-1)^{m}}{2^{2n}}\begin{pmatrix}
			2n \\
			n-m
		\end{pmatrix} \pi & [n \geq m] \hspace{2.1cm} \\
		& = 0 & [n < m] \hspace{2.1cm} \\
		\tag*{Би (40)(16), ГХ2 (332)(12b)}
	\end{align*}
	\begin{align*}
		\makebox[0pt]{$13.^{7}$} \qquad \quad \int_{0}^{\pi} \sin^{2n+1}x&\cos2mx \,dx = \\
		& \hspace{.2cm} = 2\int_{0}^{\pi/2} \sin^{2n+1}x\cos2mx \,dx = \frac{(-1)^{m}2^{n+1}n!(2n+1)!!}{(2m-2n-3)!!(2m+2n+1)!!} \hspace{.9cm} & [n \leq m - 1] \\
		& \hspace{.2cm} = \frac{(-1)^{n+1}2^{n+1}n!(2m-2n+3)!!(2n+1)!!}{(2m+2n+1)!!} \hspace{.9cm} & [n < m - 1] \\
		\tag*{ГХ2 (332)(12c)}
	\end{align*}
	\begin{align*}
		& \makebox[0pt]{14.} \qquad \quad \int_{0}^{\pi/2} \cos^{\nu-2}x\sin\nu x \,dx = \frac{1}{\nu - 1} \hspace{2.3cm} & [Re \enspace\nu > 1] \tag*{ГХ2 (332)(16с), Фх II 152}
	\end{align*}
	\begin{align*}
		\makebox[0pt]{15.} \qquad \quad \int_{0}^{\pi} \cos^{m}\sin nx \,dx & = [1-(-1)^{m+n}] \int_{0}^{\pi/2} \cos^{m}x\sin nx \,dx = \\
		& = [1 - (-1)^{m+n}]\Biggl\{\sum_{k=0}^{r-1} \frac{m!}{(m-k)!}\frac{(m+n-2k-2)!!}{(m+n)!!} + s\frac{m!(n - m - 2)!!}{(m + n)!!}\Biggr\} \\
	\end{align*}
	\vspace{-1.5cm}
	\begin{align*}
		& \qquad \quad \hspace{1cm} \left[\ r = \begin{cases}
			m, & \text{если $m \leq n$} \\
			n, & \text{если $m \geq n$}
		\end{cases}, \quad s = 
		\begin{cases}
			2, & \text{если $n - m = 4l + 2 > 0$}\\
			1, & \text{если $n - m = 2l + 1 > 0$}\\
			0, & \text{если $n - m = 4l$ или $n - m < 0$}
		\end{cases}\right]\ \tag*{ГХ2 (332)(13a)}
	\end{align*}
	\begin{align*}
		\makebox[12pt]{16.} \qquad \quad \int_{0}^{\pi/2} \cos^{n}x\sin nx \,dx = \frac{1}{2^{n+1}}\sum_{k=1}^{n}\frac{2^{k}}{k} \tag*{Фх II 153}
	\end{align*}
\end{document}