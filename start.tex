\documentclass[10pt,fleqn]{article}
\usepackage[textwidth=18cm, textheight=24cm]{geometry}

\usepackage{graphicx}

% \usepackage[condensed,math]{iwona}
\usepackage[russian,english]{babel}
\usepackage[T2C]{fontenc}
% \usepackage{uarial}
% \renewcommand{\familydefault}{\sfdefault}
% \usepackage{blindtext}

\usepackage{array}
\usepackage{tabularx}
\usepackage{ragged2e}
\usepackage{amsmath}
\usepackage{enumitem}
\usepackage{longtable}

% For russian controls
\usepackage{amssymb}

% Global vars
\newcommand\formulaGap{9.3cm}
\newcommand\reWidth{3cm}
\newcommand\rightCeilWidth{3.7cm}

\title{Степени тригонометрических функций и линейных функций}
\author{Илья Семенов}

\pagenumbering{gobble}

% Makes possible using vars
\makeatletter

% Document margins correction
\sloppy

% Add new table column styles
\newcolumntype{C}[1]{>{\centering\let\newline\\\arraybackslash\hspace{0pt}}m{#1}}
\newcolumntype{R}[1]{>{\RaggedLeft\arraybackslash}p{#1}}

\setlength{\tabcolsep}{0pt}
\renewcommand{\arraystretch}{1}

% Replace eng >= <= to rus ones
\renewcommand{\leq}{\leqslant}
\renewcommand{\geq}{\geqslant}

\begin{document}
	\begin{tabularx}{\textwidth}{m{.1\textwidth} C{.8\textwidth} R{.1\textwidth}}
		3.631 & \@title & \textbf{401} \\ 
		\hline
		\hline
	\end{tabularx}

	
	\begin{align*}
		7. \qquad \int_{0}^{\pi} x\cos(2m+1)x \,dx = 0 \tag*{ ГХ2 (332)(12a)}
	\end{align*}
	
	\begin{align*}
		8. \qquad \int_{0}^{\pi} \sin^{v-1}x\cos ax \,dx = \frac{\pi \cos\frac{a \pi}{2}}{2^{\nu-1}\nu B\left(\dfrac{\nu+a+1}{2},\dfrac{\nu-a+1}{2}\right)} & \\
		& [Re\enspace\nu > 0] \tag*{Лб V 121(68)и, Вт 337и}
	\end{align*}
	\begin{align*}
		9. \qquad \int_{0}^{\pi/2} \sin^{v-1}x\cos ax \,dx = \frac{\pi}{2^{\nu}\nu B\left(\dfrac{\nu+a+1}{2},\dfrac{\nu-a+1}{2}\right)} & \\
		& [Re\enspace\nu > 0] \tag*{ГХ2 (332)(9c)}
	\end{align*}
	\begin{align*}
		10. \qquad \int_{0}^{\pi/2} \sin^{\nu-2}x\cos\nu x \,dx = \frac{1}{\nu-1}\sin\frac{\nu\pi}{2} &
		& [Re\enspace\nu > 1] \hspace{1.3cm} \tag*{Х2 (332)(9c)}
	\end{align*}
	\begin{align*}
		11. \qquad \int_{0}^{\pi} \sin^{\nu}x\cos\nu x \,dx = \frac{\pi}{2^{\nu}}\cos\frac{\nu\pi}{2} &
		\hspace{.3cm} & [Re\enspace\nu > -1] \tag*{Лб V 121(70)и}
	\end{align*}
	\begin{align*}
		12. \qquad \int_{0}^{\pi} \sin^{2n}x\cos2mx \,dx & = 2\int_{0}^{\pi/2} \sin^{2n}x\cos2mx \,dx = \frac{(-1)^{m}}{2^{2n}}\begin{pmatrix}
			2n \\
			n-m
		\end{pmatrix} \pi & [n \geq m] \hspace{2.1cm} \\
		& = 0 & [n < m] \hspace{2.1cm}
	\end{align*}
	\begin{flushright}
		Би (40)(16), ГХ2 (332)(12b)
	\end{flushright}
\end{document}